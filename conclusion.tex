\chapter{Conclusion} \label{chapter:conclusion}
Autonomous racing serves as a compelling application to develop control algorithms because it involves scenarios with complex rules and objectives. We construct a hierarchical game-theoretic controller for autonomous racing by breaking the challenging problem into two simplified levels. The high-level component encodes the complex rules, which often include constraints over discrete parameters and variables, by constructing a discrete abstraction of the game. Using a discrete representation allows us to consider maneuvers deeper in the future and simplifies the problem for our low-level controller, which just needs to follow the provided long-term plan. We show that hierarchical game-theoretic reasoning produces behavior that is not only competitive but also safer with respect to the rules. 

Abstractly, hierarchical reasoning represents how we, as humans, make long-term decisions when taking on a challenging task. It is impossible to consider all of the details and implications for every situation along the way to complete the task. Therefore, we break the problem into a sequence of discrete checkpoints that we consider reaching along the way to achieving our final goal. 

\section{Summary of Contributions}
This report introduces the following key contributions:
\begin{enumerate}
    \item We develop a model to transform a complex head-to-head racing game with realistic safety and fairness rules into a simplified discrete game with temporal logic specifications. This representation is solved by model checking tools to produce strategies that resemble those performed by human experts.
    
    \item We use our discrete game model as a high-level planner and combine it with a pair of low-level controllers using multi-agent reinforcement learning and linear-quadratic Nash game methods to produce a pair of hierarchical controllers. The discrete model is solved in real-time using Monte Carlo tree search to estimate a strategic plan that is tracked by the low-level controllers. Our hierarchical controllers outperform other baseline controllers resembling prior works in autonomous racing research. The hierarchical structure enables them to produce and execute plans that are optimal in the long-term and mirror those seen in real-life racing.
    
    \item We extend our hierarchical control design to a team-based racing game where players have a mix of competitive and cooperative objectives. To our knowledge, this report is the first work to study this problem. We show that our hierarchical controller scales to the even more complex problem and outperforms the baseline methods adapted to the team-based racing game. Again, our model produces behavior that mimics strategies executed by expert human drivers.
\end{enumerate}
\section{Future Extensions}
Future extensions of this work should introduce additional high-level and low-level planners. Examples of additional low-level controllers include time-varying linear-quadratic approximations or iterative best response methods. With a larger collection of control options, one might investigate policy-switching hierarchical controllers where we switch between various high and low-level controllers depending on the state of the game. For example, we noticed in our experiments that the LQNG-based controllers are reliable in situations where there are no nearby opponents avoiding any risk of issues arising from using a black-box MARL-based controller. Furthermore, if there are no other players nearby, then the high-level control can also switch to following the fixed, optimal line as all other racing trajectories would be sub-optimal. Deciding when to make such switches would possibly require an additional layer of hierarchy.

Lastly, our hierarchical control design can be extended to other multi-agent systems applications where there exist complex rules such as energy grid systems or air traffic control. Constructing a discrete high-level game allows for natural encoding of the complex constraints, often involving discrete components, to find an approximate solution that can warm start a more precise low-level planner.